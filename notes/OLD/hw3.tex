\documentclass[12pt]{article}

%AMS-TeX packages
\usepackage{amssymb,amsmath,amsthm}
%geometry (sets margin) and other useful packages
\usepackage[margin=1.25in]{geometry}
\usepackage{tikz}
\usepackage{tikz-cd}
\usepackage{graphicx,ctable,booktabs}
\usepackage{mathpartir}

\usepackage[sort&compress,square,comma,authoryear]{natbib}
\bibliographystyle{plainnat}

%
%Redefining sections as problems
%
\makeatletter

\theoremstyle{definition}
\newtheorem{lemma}{Lemma}
\newtheorem{checklaw}{Check}[section]

\newenvironment{problem}{\@startsection
       {section}
       {1}
       {-.2em}
       {-3.5ex plus -1ex minus -.2ex}
       {2.3ex plus .2ex}
       {\pagebreak[3]%forces pagebreak when space is small; use \eject for better results
       \large\bf\noindent{Problem }
       }
       }
       {%\vspace{1ex}\begin{center} \rule{0.3\linewidth}{.3pt}\end{center}}
       \begin{center}\large\bf \ldots\ldots\ldots\end{center}}
\makeatother


%
%Fancy-header package to modify header/page numbering
%
\usepackage{fancyhdr}
\pagestyle{fancy}
%\addtolength{\headwidth}{\marginparsep} %these change header-rule width
%\addtolength{\headwidth}{\marginparwidth}
%% \lhead{Problem \thesection}
\chead{}
\rhead{\thepage}
\lfoot{\small\scshape EECS 598: Category Theory}
\cfoot{}
\rfoot{\footnotesize PS 1}
\renewcommand{\headrulewidth}{.3pt}
\renewcommand{\footrulewidth}{.3pt}
\setlength\voffset{-0.25in}
\setlength\textheight{648pt}

%%%%%%%%%%%%%%%%%%%%%%%%%%%%%%%%%%%%%%%%%%%%%%%

%
%Contents of problem set
%

\newcommand{\meet}{\wedge}
\newcommand{\join}{\vee}
\newcommand{\iplmeets}{\textrm{IPL}(\top,\meet)}
\newcommand{\iplneg}{\textrm{IPL}(\top,\meet,\supset)}
\newcommand{\downset}{\mathcal P_{\downarrow}}
\newcommand{\down}{{\downarrow}}

\newcommand{\Set}{\textrm{Set}}
\newcommand{\casePlus}[5]{\textrm{case}_{+}\,{#1}\{\sigma_1{#2}\to {#3}|\sigma_2{#4}\to {#5}\}}
\newcommand{\caseZero}[1]{\textrm{case}_0\,{#1}\{\}}
\newcommand{\id}{\textrm{id}}
\newcommand{\lfpt}{\Sigma_{\textrm{lfpt}}}

\newcommand{\cat}{\mathcal}

\begin{document}

\title{Problem Set 3: Functoriality and Naturality}

\author{Eric Bond}
\maketitle


\begin{problem}{Bifunctors}

    \subsection{}
    It s.t.s. an isomorphism $\textbf{Joint} \cong \textbf{Sep}$ in Set. Note that equality of functors is determined by equality of the action on objects and morphisms.
    \subsubsection*{Sep $\to$ Joint}
    Given separate $F : \cat C,\cat D \to \cat E$, define joint $G : \cat C, \cat D, \to \cat E$.
    
    \begin{align*}
        &G_0 := F_0 \\
        &G_1 (f : \cat C[c,c'],g : \cat D[d,d']) := F_l(f) \circ F_r(g)
    \end{align*}

    check
    \begin{align*}
        G(id_c,id_d) &= F_l(id_c) \circ F_r(id_d)\\
        &= id_{F(c,d) \circ id_{F(c,d)}}\\
        &= id_{F(c,d)}
    \end{align*}
    \begin{align*}
        G(f' \circ f,g' \circ g) &= F_l(f' \circ f) \circ F_r (g' \circ g)\\
        &=F_l(f') \circ F_l(f) \circ F_r(g') \circ F_r(g) \\
        &=F_l(f') \circ F_r(g') \circ F_l(f) \circ F_r(g) \\
        &=G(f',g') \circ G(f,g)
    \end{align*}


    \subsubsection*{Joint $\to$ Sep}
Given joint $F : \cat C,\cat D \to \cat E$, define sep $G : \cat C, \cat D, \to \cat E$.

    \begin{align*}
        &G_0 := F_0 \\
        &G_l (f : \cat C[c,c']) := F_1(f,id_d)\\
        &G_r (g : \cat D[c,d']) := F_1(id_c,g)
    \end{align*}

check

\begin{align*}
    G_l(id_c) &= F(id_c,id_d) = id_{F(c,d)}\\
    G_l(f' \circ f) &= F(f' \circ f,id_d)\\
    &= F(f' \circ f,id_d \circ id_d)\\
    &= F(f',id_d) \circ F(f,id_d)\\
    &= G_l (f') \circ G_l(f)
\end{align*}

same for $G_r$ by symmetry.

\begin{align*}
    G_l(f) \circ G_r(g) &= F(f,id_{d'}) \circ F(id_c,g)\\
    &=F(f \circ id_c, id_{d'} \circ g)\\
    &=F(f,g)\\
    &=F(id_{c'}\circ f, g \circ id_d)\\
    &=F(id_{c'},g) \circ F(f,id_d)\\
    &=G_r(g) \circ G_l(f)
\end{align*}

\subsubsection*{Iso}

Given joint $F : \cat C, \cat D \to \cat E$, we have $F_0 \mapsto F_0$. For the action on morphisms, we have 
\[
  F(f,g) \mapsto F(f,id_d) \circ F(id_c,g) = F(f,g)
\]
Given sep $F : \cat C, \cat D \to \cat E$, we have $F_0 \mapsto F_0$. For the separate action on morphisms, we have
\begin{align*}
    F_l(f),F_r(g)  &\mapsto F_l(f) \circ F_r(id_d), F_l(id_c) \circ F_r(g) \\
    &= F_l(f) \circ id_{F(c,d)},  id_{F(c,d')} \circ F_r(g)\\
    &= F_l(f), F_r(g)
\end{align*}


\subsection{}
It s.t.s. an isomorphism $\textbf{Joint} \cong \textbf{Prod}$ in Set. Note that equality of functors is determined by equality of the action on objects and morphisms.
\subsubsection*{Joint $\to$ Prod}
Given joint $F : \cat C,\cat D \to \cat E$, define $G : \cat C \times \cat D \to \cat E$.
\begin{align*}
    G_0((c,d)) &:= F_0(c,d)\\
    G_1((f : \cat C[c,c'],g : \cat D[d,d'])) & := F_1(f,g)
\end{align*}

check 

\begin{align*}
    G((id_c,id_d)) &= F(id_c,id_d) = id_{F(c,d)} = id_{G((c,d))}\\
    G((f' ,g') \circ (f, g)) &= G((f' \circ f,g' \circ g))\\
    &=F(f' \circ f, g' \circ g)\\
    &=F(f',g') \circ F(f,g)\\
    &=G((f',g')) \circ G((f,g))
\end{align*}

\subsubsection*{Prod $\to$ Joint}
Given $F : \cat C \times \cat D \to \cat E$, define $G : \cat C ,\cat D \to \cat E$.
\begin{align*}
    G_0(c,d) &:= F_0((c,d))\\
    G_1(f,g) &:= F_1((f,g)) 
\end{align*}
Check
\begin{align*}
 G(id_c,id_d) &= F((id_c,id_d)) = id_{F((c,d))} = id_{G(c,d)} \\
 G(f' \circ f, g' \circ g) &= F((f' \circ f,g' \circ g))\\
 &=F((f',g')) \circ F((f,g)) \\
 &=G(f',g') \circ G(f,g)
\end{align*}

\subsubsection*{Iso}
Given $F : \cat C, \cat D \to \cat E$,
\[
    F_0(c,d) \mapsto F_0(c,d)
\]
and 
\[
    F_1(f,g) \mapsto F_1(f,g)   
\]
Given $F : \cat C \times \cat D \to \cat E$,
\[
    F_0(c,d) \mapsto F_0(c,d)    
\]
and 
\[
    F_1(f,g) \mapsto F_1(f,g)    
\]

\subsection{}
It s.t.s. an isomorphism $\textbf{Sep} \cong \textbf{Fun}$ in Set. Note that equality of functors is determined by equality of the action on objects and morphisms.

\subsubsection*{Sep $\to$ Fun}
Given $F : \cat C , \cat D \to \cat E$, define $G : \cat C \to \cat E^{\cat D}$.
\begin{align*} 
    &G_0(c) := H\\
    &\quad \text{where}\\
    &\quad \quad H : \cat D \to \cat E\\
    &\quad \quad H_0(d) := F_0(c,d)\\
    &\quad \quad H_1(g : \cat D[d,d']) := F_r(g)\\
    &G_1(f : \cat C[c,c']) := \alpha\\
    &\quad \text{where}\\
    &\quad \quad \alpha : G_0(c) \Rightarrow G_0(c')\\
    &\quad \quad \alpha(d) : \cat E[F_0(c,d),F_0(c',d)]\\
    &\quad \quad \alpha(d) := F_l(f)
\end{align*}
First, check that any $G_0(c)= H$ obeys the functor laws.
\begin{align*} 
    H(id_d) &= F_r(id_d) = id_{F(c,d)} = id_{H(d)}\\
    H(g' \circ g) &= F_r(g' \circ g)\\
    &=F_r(g') \circ F_r(g)\\
    &=H(g') \circ H(g)
\end{align*}
Second, check that any $G_1(f : \cat C[c,c']) = \alpha$ is natural. Let $F^c := G_0(c)$ and $F^{c'} := G_0(c')$.
Given any $g : \cat D[d,d']$,
\[
\begin{tikzcd}
F^c(d) \arrow[r, "F^c(g)"] \arrow[d, "\alpha_d"'] & F^c(d') \arrow[d, "\alpha_{d'}"] \\
F^{c'}(d) \arrow[r, "F^{c'}(g)"'] & F^{c'}(d')
\end{tikzcd}
\]
If we unfold all the definitions, this square is just the commutativity condition for $F$
\[
\begin{tikzcd}
F(c,d) \arrow[r, "F_r(g)"] \arrow[d, "F_l(f)"'] & F(c,d') \arrow[d, "F_l(f)"] \\
F(c',d) \arrow[r, "F_r(g)"'] & F(c',d')
\end{tikzcd}
\]
Now we need to check that $G$ is actually a functor. $G(id_c)= \alpha : F(c,\_) \Rightarrow F(c,\_)$ is a natural transformation where the component maps are $\alpha(d):= F_l(id_c) = id_{F(c,d)}$, which is the identity natural transformation. $G(f' \circ f)= \alpha$ is a natural transformation with component maps $\alpha(d) := F_l(f' \circ f)$. This is equal to vertical composition $\alpha' \circ \alpha$ of natural transformations with components $\alpha(d) := F_l(f)$ and $\alpha'(d) := F_l(f')$. So $G(f' \circ f) = G(f') \circ G(f)$.


\subsubsection*{Fun $\to$ Sep}
Given $F :\cat C \to \cat E^{\cat D} $, define $G :\cat C , \cat D \to \cat E $.
\begin{align*}
    &G_0(c,d) := (F_0(c))_0(d)\\
    &G_l(d)(f: \cat C[c,c']) : \cat E[G_0(c,d),G_0(c',d)]\\
    &G_l(d)(f)  := F_1(f).\alpha(d)\\
    &G_r(c)(g : \cat D[d,d'] : \cat E[G_0(c,d),G_0(c,d')])\\
    &G_r(c)(g) := (F_0(c))_1(g)
\end{align*}
Now check the laws.\\
\[
    G_l(d)(id_c) = F_1(id_c).\alpha(d)= id_{F_0(c)}.\alpha(d) = id_{(F_0(c))_0(d)} = id_{G_0(c,d)}
\]
\begin{align*} 
    G_l(d)(f' \circ f) &= F_1(f' \circ f).\alpha(d) \\
    &= (F_1(f') \circ F_1(f)).\alpha(d)\\
    &= F_1(f').\alpha(d) \circ F_1(f).\alpha(d)\\
    &= G_l(d)(f') \circ G_l(d)(f)
\end{align*}

\begin{align*} 
    G_l(d')(f) \circ G_r(c)(g) &= F_1(f).\alpha(d) \circ (F_0(c))_1(g)\\
    &\text{wiskering something something}
\end{align*}

\end{problem}



\begin{problem}{Product Functor}
    \subsection{}
    \begin{align*}
        &\times : \cat C \times \cat C \to \cat C\\
        &\times_0(c,c') := c \times c'\\
        &\times_1(f,g) : \cat C[c_1 \times c_2, c_3 \times c_4]\\
        &\times_1(f,g) := (f \circ \pi_1,g \circ \pi_2)
    \end{align*}
The action on morphisms is given by the universal "mapping in" property of the product in the codomain.
We have to show $\times(id_c,id_{c'})=id_{c\times c'}$ but this holds by uniqueness of maps into the product. Similarly $\times(f'\circ f, g' \circ g)=\times(f',g') \circ \times(f,g)$ holds by uniqueness of maps into the product.

\subsection{}
We have 
\begin{align*}
    &(\Pi_1)_0(c,c') := c\\
    &(\Pi_1)_1(f,g) := f
\end{align*}
Construct $\alpha : \times \Rightarrow \Pi_1$ as,
\begin{align*}
    &\alpha((c, c')) : \cat C[c \times c', c ]\\
    &\alpha((c,c')) := \pi_1 
\end{align*}
$\forall (f,g) : \cat C\times \cat C[(c_1,c_2),(c_3,c_4)],$ naturality is exactly the commuting requirement of the unique morphism into the product $c_3,c_4$.
\[
\begin{tikzcd}
    c_1 \times c_2 \arrow[r, "{(f\circ \pi_1,g \circ \pi_2)}"] \arrow[d, "\pi_1"'] & c_3 \times c_4 \arrow[d, "\pi_1"] \\
    c_1 \arrow[r, "f"'] & c_3
    \end{tikzcd}
\]



\end{problem}


\begin{problem}{Theorems for Free, Naturally}

\subsection{}
A natural transformation $\alpha : Id \Rightarrow Id$ consists of family of component maps $\forall X \in \textbf{Set}_0, \alpha(X):\textbf{Set}[X,X]$ such that $\forall f : \textbf{Set}[X,Y]$,
\[
    \begin{tikzcd}
        X \arrow[r, "f"] \arrow[d, "\alpha(X)"'] & Y \arrow[d, "\alpha(Y)"] \\
        X \arrow[r, "f"'] & Y
        \end{tikzcd}
\]
Consider the case where $f : \textbf{Set}[1,X]$.
\[
    \begin{tikzcd}
        1 \arrow[r, "f"] \arrow[d,dashed, "!"'] & X \arrow[d, "\alpha(X)"] \\
        1 \arrow[r, "f"'] & X
        \end{tikzcd}
\]
The map $\alpha(1)$ is uniquely determined. The map $f$ amounts to picking an element $x : X$. In order for this diagram to commute, it must be the case that $\alpha(X) := id_X$ as we can vary $f$ to choose any element in $X$. This situation applies to any other choice of set $A$ in place of $X$. Therefore, the component maps must be defined uniformly in choice of $X$ and the only choice is the identity map.

\subsection{}
A natural transformation $\alpha : \times \Rightarrow \times'$ consists of family of component maps $\forall (A,A') \in (\textbf{Set} \times \textbf{Set})_0, \alpha((A,A')):\textbf{Set} \times \textbf{Set}[A \times A', A' \times A]$ such that $\forall (f,g) : \textbf{Set} \times \textbf{Set}[A \times A',B \times B']$,
\[
    \begin{tikzcd}
        A\times A' \arrow[r, "{f,g}"] \arrow[d, "{\alpha(A,A')}"'] & B\times B' \arrow[d, "{\alpha(B,B')}"] \\
        A' \times A \arrow[r, "{g,f}"'] & B'\times B
        \end{tikzcd}
\]
Consider the case where $f : \textbf{Set}[\{\bullet\},B],g : \textbf{Set}[\{\circ\},B']$
\[
    \begin{tikzcd}
        \{\bullet\} \times \{\circ\} \arrow[r, "{f,g}"] \arrow[d,dashed, "!=id"'] & B\times B' \arrow[d, "{\alpha(B,B')}"] \\
        \{\circ\} \times \{\bullet\} \arrow[r, "{g,f}"'] & B'\times B
        \end{tikzcd}
\]
Pointwise, naturality states that $\alpha(B,B')(b,b')=(b',b)$, forcing the definition of $\alpha$ to be swap.

\subsection{}
A natural transformation $\alpha : Id_{\textbf{Set*}} \Rightarrow Id_{\textbf{Set*}}$ consists of family of component maps $\forall A^* \in \textbf{Set*}_0, \alpha(A^*):\textbf{Set*}[A^*,A^*]$ such that $\forall f : \textbf{Set*}[A^*,B^*]$,

\[
    \begin{tikzcd}
        A^* \arrow[r, "f"] \arrow[d, "\alpha(A*)"'] & B^* \arrow[d, "\alpha(B^*)"] \\
        A^* \arrow[r, "f"'] & B^*
        \end{tikzcd}
\]
One possible natural transformation maps every element of a pointed set to the base point.
\[
  \alpha(A^*)(\_):=  a_0  
\]
Naturality can be demonstrated pointwise. 
\[
    a \mapsto a_0 \mapsto (f(a_0)=b_0) = a \mapsto f(a) \mapsto b_0
\]
Another possibility is a natural transformation with the identity function as the component map.
\[
\alpha(A*)(a) := a    
\]
Pointwise..
\[
    a \mapsto a \mapsto f(a) = a \mapsto f(a) \mapsto f(a)
\]
Consider
\[
    \begin{tikzcd}
        \{a_0,a_1\} \arrow[r, "f"] \arrow[d, "{\alpha(\{a_0,a_1\})}"'] & B^* \arrow[d, "\alpha(B^*)"] \\
        \{a_0,a_1\} \arrow[r, "f"'] & B^*
        \end{tikzcd}
\]
All morphisms in this category must preserve the base point. The only choice in defining $f$ and $\alpha(\{a_0,a_1\})$ is where to map $a_1$. Assume $\alpha(a_1) := a_1$. The naturality condition is 
\[
  \alpha(B^*)(f(a_1))= f(a_1)  
\] 
which forces $\alpha(B^*)$ to behave like the identity function. Assume $\alpha(a_1):= a_0$. The naturality condition is 
\[
    \alpha(B^*)(f(a_1)) = b_0
\]
which forces $\alpha(B^*)$ to map any $b : B^*$ to $b_0$. These are our two choices of natural transformations above.
\newpage
  The naturality property of a natural transformation is such a strong
  condition that sometimes we can characterize all natural
  transformations between two fixed functors, and in many examples
  there are only finitely many.
  
  This has direct applications to programming. The reason is that in a
  pure polymorphic functional language, given type constructors $F$
  and $G$ that are functorial, all functions $F(X) \to G(X)$ that are
  polymorphic in $X$ denote natural transformations! Phil Wadler,
  building on John Reynold's theory of parametricity called these
  ``theorems for free'': just from the type of a polymorphic function,
  the naturality property gives you properties that hold for every
  function of that type (\cite{wadler,reynolds}).
  \begin{enumerate}
  \item Define a natural transformation from $\id_{\Set}$ to
    $\id_{\Set}$ and prove that it is the only such natural
    transformation.

  \item Let $\times : \Set \times \Set \to \Set$ be the functor you
    deifned in the previous problem and let $\times' : \Set \times
    \Set \to \Set$ be the functor with the arguments swapped $A
    \times' B = B \times A$.

    Define a natural transformation from $\times$ to $\times'$ and
    show that it is the only such natural transformation.

  \item Recall the category of pointed sets $\Set_*$ is defined as follows:
    \begin{itemize}
    \item Objects are pairs of a set $X$ and a ``basepoint'' $x_0 \in X$.
    \item A morphism from $(X,x_0)$ to $(Y, y_0)$ is a
      \emph{base-point-preserving} function, i.e., a function $f : X
      \to Y$ such that $f(x_0) = y_0$. Identity and composition are
      simply identity and composition of functions.
    \end{itemize}

    Define two different natural transformations from
    $\id_{\Set_*}$ to $\id_{\Set_*}$ and prove that these
    are the only two such natural transformations.
  \end{enumerate}
\end{problem}
\newpage
\bibliography{cats}

\end{document}