\documentclass{article}
\usepackage{graphicx} % Required for inserting images
\usepackage{simplebnf}
\usepackage{bussproofs}
\usepackage{amsmath}
\usepackage{amssymb}
\usepackage[dvipsnames]{xcolor}
\usepackage{stmaryrd}
\usepackage{comment}
\usepackage{bm}

\begin{document}
%\usepackage{xcolor}

\newcommand{\den}[1]{\llbracket #1 \rrbracket}
\newcommand{\blue}[1]{\textcolor{blue}{#1}}
\newcommand{\red}[1]{\textcolor{red}{#1}}
\newcommand{\sep}{\mathrel{-\mkern-6mu*}}
\newcommand{\thunk}[1]{\textrm{thunk }#1}
\newcommand{\injj}[2]{\textrm{inj}_{#1}#2}
\newcommand{\err}{\mho}
\newcommand{\print}[1]{\textrm{print }#1}
\newcommand{\force}[1]{\textrm{force }#1}
\newcommand{\ret}[1]{\textrm{ret }#1}
\newcommand{\bind}[3]{#1 \leftarrow #2 ; #3}
\newcommand{\newcase}[3]{\textrm{newcase}_{#1} \; #2 ; #3}
\newcommand{\match}[5]{\textrm{match }#1 \textrm{ with }#2 \;\{#3 . #4 | #5\}}
\newcommand{\unpack}[4]{\textrm{unpack }(#1,#2) = #3 ; #4}
\newcommand{\lett}[4]{\textrm{let }(#1,#2) = #3 ; #4}
\newcommand{\lets}[4]{\textrm{let }(#1*#2) = #3 ; #4}
\newcommand{\ite}[3]{\textrm{if }#1 \textrm{ then }#2 \textrm{ else }#3}
\newcommand{\at}{\textrm{@}}
\newcommand{\ttt}{\textrm{tt}}
\newcommand{\tru}{\textrm{true}}
\newcommand{\fsl}{\textrm{false}}
\newcommand{\pworld}{\widehat{\mathbf{World}}}
\newcommand{\world}{{\mathbf{World}}}
\newcommand{\calculus}{\operatorname{-calculus}}


\section{Object Language}
A simply typed CBPV language.
\subsection{Raw Terms}
\begin{bnf}
    $A$ : Value Types ::= Unit
    | $A \times A$
    | $U \underline{B}$
    ;;
    $\underline{B}$ : Computation Types ::= $A \rightarrow \underline{B}$
    | $F A$
    ;;
    $V$ : Values ::= $x$
    | $tt$
    | $(V , V)$
    | thunk $M$
    ;;
    $M$ : Computations ::= 
     $\lambda x \colon A . M$
    | $M V$
    | ret $V$
    | force $V$
    | $x \leftarrow M ;M'$
    | let $x$ = $V ; M'$
    | let ($x , y$) = $V ; M$
    ;;
    $\Gamma$ : Value Context ::= $\cdot$
    | $\Gamma , x \colon A$ 
    ;;
\end{bnf}
\subsection{Typed Terms}
\begin{prooftree}
    \AxiomC{}
    \RightLabel{Var}
    \UnaryInfC{$\Gamma, x \colon A \vdash_v x : A$}
\end{prooftree}

\begin{prooftree}
    \AxiomC{}
    \RightLabel{I-Unit}
    \UnaryInfC{$\Gamma \vdash_v tt : $ Unit}
\end{prooftree}


\begin{prooftree}
    \AxiomC{$\Gamma \vdash_v t : A$}
    \AxiomC{$\Gamma \vdash_v t : A'$}
    \RightLabel{I-$\times$}
    \BinaryInfC{$\Gamma \vdash_v  (t , u): A \times A' $ }
\end{prooftree}


\begin{prooftree}
    \AxiomC{$\Gamma \vdash_c M : \underline{B}$}
    \RightLabel{}
    \UnaryInfC{$\Gamma \vdash_v \textrm{thunk } M : U \underline{B}$}
\end{prooftree}


\begin{prooftree}
    \AxiomC{$\Gamma , x : A \vdash_c M : \underline{B}$}
    \RightLabel{I - $\rightarrow$}
    \UnaryInfC{$\Gamma \vdash_c (\lambda (x : A) . M) : A \rightarrow \underline{B}$}
\end{prooftree}


\begin{prooftree}
    \AxiomC{$\Gamma \vdash_c M : A \rightarrow \underline{B}$}
    \AxiomC{$\Gamma \vdash_v V : A$}
    \RightLabel{E-$\rightarrow$}
    \BinaryInfC{$\Gamma \vdash_c : M V : \underline{B}$}
\end{prooftree}

\begin{prooftree}
    \AxiomC{$\Gamma \vdash_v V : A$}
    \RightLabel{}
    \UnaryInfC{$\Gamma \vdash_c \textrm{ret } V : F A$}
\end{prooftree}

\begin{prooftree}
    \AxiomC{$\Gamma \vdash_v V : U \underline{B}$}
    \RightLabel{}
    \UnaryInfC{$\Gamma \vdash_c \textrm{force } V : \underline{B}$}
\end{prooftree}

\begin{prooftree}
    \AxiomC{$\Gamma \vdash_c M : F A$}
    \AxiomC{$\Gamma , x : A \vdash_c N : \underline{B}$}
    \RightLabel{}
    \BinaryInfC{$\Gamma \vdash_c x \leftarrow M ; N: \underline{B}$}
\end{prooftree}


\begin{prooftree}
    \AxiomC{$\Gamma \vdash_v V : A$}
    \AxiomC{$\Gamma , x : A \vdash_c N : \underline{B}$}
    \RightLabel{}
    \BinaryInfC{$\Gamma \vdash_c \textrm{let x } = V ; N : \underline{B}$}
\end{prooftree}

\begin{prooftree}
    \AxiomC{$\Gamma \vdash_v V : A \times A'$}
    \AxiomC{$\Gamma , x : A, y : A' \vdash_c M : \underline{B}$}
    \RightLabel{E-$\times$}
    \BinaryInfC{$\Gamma \vdash_c \textrm{let (x,y) } = V ; M : \underline{B}$}
\end{prooftree}

\section{Simple Logic}
The usual presentation of a logic except that we make a distinction between value and 
\subsection{Formation Rules}
\subsubsection{Value Fragment}
Judgments: 
Value Propositions are over a value context.
\[
  \Gamma \vdash \phi \textrm{ VProp}  
\]
 Connectives\footnote{Focusing on just a few connectives for the moment}
 \footnote{should there be a logical equivalent to $U \underline{B}$ ($U \underline{\phi}$) here?}: 
 \[
   \phi := \top | \phi \land \phi | \phi \implies \phi 
 \]
\begin{prooftree}
    \AxiomC{}
    \UnaryInfC{$\Gamma \vdash \top  \textrm{ VProp}$}
\end{prooftree}

\begin{prooftree}
    \AxiomC{$\Gamma \vdash \phi \textrm{ VProp}$}
    \AxiomC{$\Gamma \vdash \psi \textrm{ VProp}$}
    \BinaryInfC{$\Gamma \vdash \phi \land \psi  \textrm{ VProp}$}
\end{prooftree}

\begin{prooftree}
    \AxiomC{$\Gamma \vdash \phi \textrm{ VProp}$}
    \AxiomC{$\Gamma \vdash \psi \textrm{ VProp}$}
    \BinaryInfC{$\Gamma \vdash \phi \implies \psi  \textrm{ VProp}$}
\end{prooftree}
\subsubsection{Computation Fragment}
Judgments: 
Computation Propositions are over a value context and a stoup.
\[
  \Gamma ; \Delta \vdash \underline{\phi} \textrm{ CProp}  
\]
In our intended semantics, this judgment should be interpreted as a subalgebra of $\den{ \Gamma ; \Delta }$.
We need some way of combining the value object $\den{ \Gamma }$ with computation object $\den{ \Delta }$.
Maybe we can use some $\Gamma $ fold copower\footnote{This copower connective is definable in PE} of $\Delta$, $\den{ \Gamma } \oslash \den{ \Delta } $, 
which is a connective in the Enriched Effect Calculus?
\\
 Connectives: The purpose of this document is to explore this new kind of connective, the computational implication. 
 \footnote{Need weakest precondition as part of the syntax for the logical interpretation of $A \rightarrow \underline{B}$?}
 
 \[
   \underline{\phi} := \underline{\top} | \underline{\phi} \land \underline{\psi} | \blue{\phi \implies \underline{\psi}}
 \]
 \begin{prooftree}
    \AxiomC{}
    \UnaryInfC{$\Gamma ; \Delta \vdash \underline{\top}  \textrm{ CProp}$}
\end{prooftree}


\begin{prooftree}
    \AxiomC{$\Gamma ;\Delta \vdash \underline{\phi} \textrm{ CProp}$}
    \AxiomC{$\Gamma ; \Delta \vdash \underline{\psi} \textrm{ CProp}$}
    \BinaryInfC{$\Gamma ; \Delta \vdash \underline{\phi} \land \underline{\psi}  \textrm{ CProp}$}
\end{prooftree}

The computation implication formation rule, similar to the computation function type in CBPV, takes both a value proposition and a computation proposition.
\begin{prooftree}
    \AxiomC{$\Gamma \vdash \phi \textrm{ VProp}$}
    \AxiomC{$\Gamma ; \Delta \vdash \underline{\psi} \textrm{ CProp}$}
    \BinaryInfC{$\Gamma ; \Delta \vdash \phi \implies \underline{\psi}  \textrm{ CProp}$}
\end{prooftree}

\subsection{Derivation Rules}
\subsubsection{Value Derivations}
Value Derivation Judgement
\[
    \Gamma | \Phi \vdash \phi 
\]
where $\Phi$ is a conjunction of value propositions.

\begin{prooftree}
    \AxiomC{}
    \RightLabel{I-$\top$}
    \UnaryInfC{$\Gamma | \Phi \vdash \top$}
\end{prooftree}

\begin{prooftree}
    \AxiomC{$\Gamma | \Phi \vdash \phi$}
    \AxiomC{$\Gamma | \Phi \vdash \psi $}
    \RightLabel{I-$\land$}
    \BinaryInfC{$\Gamma | \Phi \vdash \phi \land \psi$}
\end{prooftree}

\begin{prooftree}
    \AxiomC{$\Gamma | \Phi \vdash \phi \land \psi$}
    \RightLabel{E1-$\land$}
    \UnaryInfC{$\Gamma | \Phi \vdash \phi $}
\end{prooftree}

\begin{prooftree}
    \AxiomC{$\Gamma | \Phi \vdash \phi \land \psi$}
    \RightLabel{E2-$\land$}
    \UnaryInfC{$\Gamma | \Phi \vdash \psi $}
\end{prooftree}

\begin{prooftree}
    \AxiomC{$\Gamma | \Phi, \phi \vdash \psi $}
    \RightLabel{I-$\implies$}
    \UnaryInfC{$\Gamma | \Phi \vdash \phi \implies \psi $}
\end{prooftree}

\begin{prooftree}
    \AxiomC{$\Gamma | \Phi \vdash \phi \implies \psi $}
    \AxiomC{$\Gamma | \Phi \vdash \phi$}
    \RightLabel{E-$\implies$}
    \BinaryInfC{$\Gamma | \Phi \vdash \psi $}
\end{prooftree}

\subsubsection{Computation Derivations}
Computation Derivation Judgment:
\[
  \Gamma ; \Delta | \Phi ; \blue{\underline{\Psi}} \vdash \underline{\phi}  
\]
If computation derivations are allowed a context of computation propositions, 
we'll need a way to combine subobjects of the value category with subobjects of the computation category\footnote{Similar to how value contexts are combined with the stop}.

\[
\oslash^*_{\Gamma ; \Delta} : Sub_V(\den{\Gamma}) \times Sub_C(\den{\Gamma} \oslash \den{\Delta}) \rightarrow Sub_C(\den{\Gamma} \oslash \den{\Delta})
\]
\[
  \den{ \Gamma ; \Delta | \Phi ; \blue{\underline{\Psi}} \vdash \underline{\phi} } : Sub_C(\Gamma \oslash \Delta)[\oslash^*(\Phi,\blue{\underline{\Psi}}), \underline{\phi}]
\]
What is this operation? (should it be adjoint to computation implication?)

\begin{prooftree}
    \AxiomC{}
    \RightLabel{I-$\top$}
    \UnaryInfC{$\Gamma ; \Delta | \Phi ; \underline{\Psi}\vdash \underline{\top}$}
\end{prooftree}

\begin{prooftree}
    \AxiomC{$\Gamma ; \Delta | \Phi ; \underline{\Psi}\vdash \underline{\phi}$}
    \AxiomC{$\Gamma ; \Delta | \Phi ; \underline{\Psi}\vdash \underline{\psi} $}
    \RightLabel{I-$\land$}
    \BinaryInfC{$\Gamma ; \Delta | \Phi ; \underline{\Psi}\vdash \underline{\phi} \land \underline{ \psi}$}
\end{prooftree}

\begin{prooftree}
    \AxiomC{$\Gamma ; \Delta | \Phi ; \underline{\Psi}\vdash \underline{\phi} \land \underline{ \psi}$}
    \RightLabel{E1-$\land$}
    \UnaryInfC{$\Gamma ; \Delta | \Phi ; \underline{\Psi}\vdash \underline{\phi} $}
\end{prooftree}

\begin{prooftree}
    \AxiomC{$\Gamma ; \Delta | \Phi ; \underline{\Psi}\vdash \underline{\phi} \land \underline{ \psi}$}
    \RightLabel{E2-$\land$}
    \UnaryInfC{$\Gamma ; \Delta | \Phi ; \underline{\Psi}\vdash \underline{ \psi}$}
\end{prooftree}

\blue{Not clear what the semantics for this connective should be.} 
Normally, the implication would be the exponential object in the category of subobjects over $\Gamma$.
Here we might be able to say something about adjointness with $\oslash$.

\begin{prooftree}
    \AxiomC{$\Gamma ; \Delta | \Phi , \phi ; \underline{\Psi}\vdash  \underline{\psi}$}
    \RightLabel{I-$\implies$}
    \UnaryInfC{$\Gamma ; \Delta | \Phi ; \underline{\Psi}\vdash \phi \implies \underline{\psi}$}
\end{prooftree}

\begin{prooftree}
    \AxiomC{$\Gamma ; \Delta | \Phi ; \underline{\Psi}\vdash \phi \implies \underline{\psi}$}
    \AxiomC{$\Gamma | \Phi \vdash \phi$}
    \RightLabel{E-$\implies$}
    \BinaryInfC{$\Gamma ; \Delta | \Phi ; \underline{\Psi}\vdash \underline{\psi}$}
\end{prooftree}

\section{Model}
\subsection{Object Language Model}
Assume we have a typical, set based, algebra model of a simply typed CBPV language consisting of a value category 
$\mathcal{V}=Set$, a monad $T$ on $\mathcal{V}$, a computation category $\mathcal{C}$ which is the category of algebras for monad $T$, 
with the usual functors $F,U$.
\subsection{Logic Model}
Starting with the interpretation of the judgments
\begin{align*}
    \den{ \Gamma \vdash \phi } &: \textrm{ object of } Sub_{\mathcal{V}}(\den{ \Gamma })\\
    \den{  \Gamma | \Phi \vdash \phi } &: Sub_{\mathcal{V}}[\den{ \Phi } , \den{ \phi }]\\
    \den{ \Gamma ; \Delta \vdash \underline{\phi}} &: \textrm{ object of } Sub_{\mathcal{C}}(\den{ \Gamma } \oslash \den{ \Delta }) \\
    \den{ \Gamma ; \Delta | \Phi ; \underline{\Psi} \vdash \underline{\phi}} &:Sub_{\mathcal{C}}(\den{ \Gamma } \oslash \den{ \Delta })[\oslash^*_{\Gamma ; \Delta}(\den{ \Phi },\den{ \underline{\Psi} }), \den{ \phi }] \\
\end{align*}
\blue{What laws should we expect $\oslash$ and $\oslash^*_{\Gamma , \Delta}$ to obey?} 
\subsubsection{Values}
Denotation of the value propositions:
\begin{align*}
    \den{\Gamma \vdash \top} &= \mathbf{1}\\
    \den{\Gamma \vdash \phi \land \psi} &= \den{\Gamma \vdash \phi} \times \den{\Gamma \vdash \psi} \\
    \den{\Gamma \vdash \phi \implies \psi} &= \den{\Gamma \vdash \psi}^{\den{\Gamma \vdash \phi}}\\
\end{align*}
Denotation of the value derivations:
\\
Unique map into the terminal object.
\begin{prooftree}
    \AxiomC{}
    \UnaryInfC{$\den{\Gamma | \Psi \vdash \top} = \;!$}
\end{prooftree}
\begin{comment}

Pullback.
\begin{prooftree}
    \AxiomC{$\den{\Gamma | \Psi \vdash \phi}= X \xhookrightarrow{\phi} \den{\Gamma}$}
    \AxiomC{$\den{\Gamma | \Psi \vdash \psi} = Y \xhookrightarrow{\psi} \den{\Gamma}$}
    \BinaryInfC{$\den{\Gamma | \Psi \vdash \phi \land \psi } = X \times_{\den{\Gamma}} Y \hookrightarrow \den{\Gamma} $}
\end{prooftree}

\begin{prooftree}
    \AxiomC{$\den{\Gamma | \Psi \vdash \phi \land \psi } = X \times_{\den{\Gamma}} Y \hookrightarrow \den{\Gamma} $}
    \UnaryInfC{$\den{\Gamma | \Psi \vdash \phi  } = ? $}
\end{prooftree}
\begin{prooftree}
    \AxiomC{$\den{\Gamma | \Psi \vdash \phi}= X \xhookrightarrow{\phi} \den{\Gamma}$}
    \AxiomC{$\den{\Gamma | \Psi \vdash \psi} = Y \xhookrightarrow{\psi} \den{\Gamma}$}
    \BinaryInfC{$\den{\Gamma | \Psi \vdash \phi \implies \psi } = \;? $}
\end{prooftree}
\end{comment}

\subsubsection{Computations}
Denotation of the computation propositions:
\begin{align*}
    \den{\Gamma ; \Delta \vdash \top} &= \mathbf{1}\\
    \den{\Gamma ; \Delta \vdash \phi \land \psi} &= \den{\Gamma ; \Delta \vdash \phi} \times \den{\Gamma ; \Delta \vdash \psi} \\
    \den{\Gamma ; \Delta \vdash \phi \implies \underline{\psi}} &= ? \\
\end{align*}
\section{Unary Relational Interpretation}
\begin{itemize}
    \item Add predicates?
    \item Why do we need a unary relational interpretaiton of types? As a way to propogate predicates over types?
    \item Do we need to add weakest precondition as a piece of syntax to describe the unary relational interpretation of $A \rightarrow \underline{B}$?
\end{itemize}

\end{document}