\documentclass{article}
\usepackage{graphicx} % Required for inserting images
\usepackage{simplebnf}
\usepackage{comment}
\usepackage{bussproofs}
\usepackage[llbracket,rrbracket]{stmaryrd}
\usepackage{amsmath}
\usepackage{amssymb}
\usepackage[dvipsnames]{xcolor}
\usepackage{cite}
\usepackage{scalerel}
\usepackage{oz}
\def\sq{\mathbin{\scalerel*{\strut\rule[-.5ex]{2ex}{2ex}}{\cdot}}}


\newcommand{\blue}[1]{\textcolor{blue}{#1}}
\newcommand\sep{\mathrel{-\mkern-6mu*}}

\title{Honey Bunches of OSum}
\author{Eric Bond}
\date{January 2024}

\begin{document}
\maketitle

\section{Untyped Syntax}
\begin{bnfgrammar}
$A$ : value types ::= $X$ 
| $\mathbf{1}$
| $A\; \times \; A$  
| $A\; * \; A$
| $U\underline{B}$
;;
$B$ : computation types ::= $A \rightarrow \underline{B}$ 
| $A \sep \underline{B}$
| $F\;A$
;;
$\Gamma$ : value typing context ::= $x\; \colon A$ 
| $\phi$ : additive unit
| $\Gamma ; \Gamma$ : additive combination (product)
| $\varphi$ : multiplicative unit
| $\Gamma \fatsemi \Gamma$ : multiplicative combination (tensor)
;;

$\Delta$ : stoup ::= $\cdot$ 
| $\bullet$
;;
\end{bnfgrammar}

\section{Typed Syntax}

\subsection{Var}
\begin{prooftree}
\AxiomC{}
\RightLabel{$Id_v$}
\UnaryInfC{$x : A \vdash x : A$}
\end{prooftree}

\begin{prooftree}
\AxiomC{}
\RightLabel{$Id_c$}
\UnaryInfC{$\Gamma | \bullet : \underline{B} \vdash \bullet : \underline{B}$}
\end{prooftree}

\subsection{One}
\begin{prooftree}
\AxiomC{}
\RightLabel{$\mathbf{1} Intro$}
\UnaryInfC{$\Gamma \vdash^v () : \mathbf{1}$}
\end{prooftree}

\begin{prooftree}
\AxiomC{$\Gamma \vdash^v M : \mathbf{1}$}
\RightLabel{$\mathbf{1} \eta$}
\UnaryInfC{$\Gamma \vdash^v ()  = M: \mathbf{1}$}
\end{prooftree}

\subsection{Product}
\begin{prooftree}
\AxiomC{$\Gamma_1  \vdash^v V_1 : A_1$}
\AxiomC{$\Gamma_2  \vdash^v V_2 : A_2$}
\RightLabel{$\times Intro$}
\BinaryInfC{$ \Gamma_1 ; \Gamma_2 \vdash^v (V_1,V_2) : A_1 \times A_2$}
\end{prooftree}

\begin{prooftree}
\AxiomC{$\Gamma \vdash^v M : A_1 \times A_2 $}
\RightLabel{$\times Elim_i$}
\UnaryInfC{$\Gamma \vdash^v \pi_i M : A_i$}
\end{prooftree}

\begin{prooftree}
\AxiomC{$\Gamma \vdash^v M_1 : A_1 $}
\AxiomC{$\Gamma \vdash^v M_2 : A_2$}
\RightLabel{$\times \beta_i$}
\BinaryInfC{$\Gamma \vdash^v \pi_i (M_1,M_2)  = M_i : A_i$}
\end{prooftree}

\begin{prooftree}
\AxiomC{$\Gamma \vdash^v M : A_1 \times A_2 $}
\RightLabel{$\times \eta$}
\UnaryInfC{$\Gamma \vdash^v M = (\pi_1 M , \pi_2 M) : A_1 \times A_2$}
\end{prooftree}

\subsection{Sep Product}
\begin{prooftree}
\AxiomC{$\Gamma_1  \vdash^v V_1 : A_1$}
\AxiomC{$\Gamma_2  \vdash^v V_2 : A_2$}
\RightLabel{$* Intro$}
\BinaryInfC{$ \Gamma_1 \fatsemi \Gamma_2 \vdash^v V_1 * V_2 : A_1 * A_2$}
\end{prooftree}

\begin{prooftree}
\AxiomC{$\Gamma_1(x : A \fatsemi y : B) \vdash^v N : C $}
\AxiomC{$\Gamma_2 \vdash^v M : A * B$}
\RightLabel{$*Elim$}
\BinaryInfC{$\Gamma_1(\Gamma_2) \vdash^v \mathbf{let} (x,y) = M \; \mathbf{in} \; N : C$}
\end{prooftree}


\begin{prooftree}
\AxiomC{$\Gamma_1 \vdash^v M_1 : A$}
\AxiomC{$\Gamma_2 \vdash^v M_2 : B $}
\AxiomC{$\Gamma_3(x : A \fatsemi y : B) \vdash^v N : C $}
\RightLabel{$* \beta$}\footnote{beta eta from page 21 of \cite{Semantics-Proof-Theory-Bunched-Implications}}
\TrinaryInfC{$\blue{\Gamma?} \vdash^v (\mathbf{let} (x,y) = (M_1 * M_2) \; \mathbf{in}\; N) = N[M_1/x,M_2/y] : C$}
\end{prooftree}

\begin{prooftree}
\AxiomC{$\Gamma \vdash^v  M : A_1 * A_2 $}
\RightLabel{$* \eta $}
\UnaryInfC{$\Gamma \vdash^v (\mathbf{let} (x,y) = M\; \mathbf{in}\; x * y) = M : A_1 * A_2$}
\end{prooftree}


\subsection{U}

\begin{prooftree}
\AxiomC{$\Gamma | \cdot \vdash^c M : \underline{B}$}
\RightLabel{$\mathbf{tf} \; Intro$}
\UnaryInfC{$\Gamma \vdash^v \mathbf{thunk} M : U \underline{B}$}
\end{prooftree}

\begin{prooftree}
\AxiomC{$\Gamma \vdash^v V : U \underline{B}$}
\RightLabel{$\mathbf{tf} \; Elim$}
\UnaryInfC{$\Gamma | \cdot \vdash^c \mathbf{force} V : \underline{B}$}
\end{prooftree}

\begin{prooftree}
\AxiomC{$\Gamma | \cdot \vdash^c M  : \underline{B}$}
\RightLabel{$\mathbf{tf} \; \beta$}
\UnaryInfC{$\Gamma | \cdot \vdash^c \mathbf{force}\; (\mathbf{thunk}\; M) = M : \underline{B}$}
\end{prooftree}

\begin{prooftree}
\AxiomC{$\Gamma \vdash^v V : U \underline{B}$}
\RightLabel{$\mathbf{tf}\; \eta$}
\UnaryInfC{$\Gamma \vdash^v V = \mathbf{thunk}\; (\mathbf{force}\; V) : U\underline{B}$}
\end{prooftree}


\subsection{"Normal" functions}
This is where things get tricky..

\begin{prooftree}
\AxiomC{$\Gamma ; (x : A) | \underline{\Delta} \vdash^c  M : \underline{B}$}
\RightLabel{$\rightarrow Intro$}
\UnaryInfC{$\Gamma | \underline{\Delta}\vdash^c (\lambda x : A .\; M) : A \rightarrow \underline{B}$}
\end{prooftree}

\begin{prooftree}
\AxiomC{$\Gamma_1 | \underline{\Delta} \vdash^c  M : A \rightarrow \underline{B} $}
\AxiomC{$\Gamma_2 \vdash^v V : A$}
\RightLabel{$\rightarrow Elim$}
\BinaryInfC{$\Gamma_1 ; \Gamma_2 | \underline{\Delta} \vdash^c M V : \underline{B}$}
\end{prooftree}

\begin{prooftree}
\AxiomC{$\Gamma_1 ; (x : A) | \underline{\Delta} \vdash^c  M : \underline{B}$}
\AxiomC{$\Gamma_2  \vdash^v N : A$}
\RightLabel{$\rightarrow \beta$}
\BinaryInfC{$\Gamma_1 ; \Gamma_2 | \underline{\Delta} \vdash^c (\lambda x : A.\; M) N = M[N/x] : \underline{B}$}
\end{prooftree}

\begin{prooftree}
\AxiomC{$\Gamma ; (x : A) | \underline{\Delta} \vdash^c  M : \underline{B}$}
\AxiomC{$x \notin FV(M)$}
\RightLabel{$\rightarrow \eta$}
\BinaryInfC{$\Gamma | \underline{\Delta} \vdash^c (\lambda x : A.\; M x) = M : A \rightarrow \underline{B}$}
\end{prooftree}




\subsection{Wand}
This is the same as $\rightarrow$, just "alpha renamed" symbols ($\fatsemi, \textit{@},\alpha, \sep$).
\begin{prooftree}
\AxiomC{$\Gamma \fatsemi (x : A) | \underline{\Delta} \vdash^c  M : \underline{B}$}
\RightLabel{$\sep Intro$}
\UnaryInfC{$\Gamma | \underline{\Delta}\vdash^c (\alpha x : A .\; M) : A \sep \underline{B}$}
\end{prooftree}

\begin{prooftree}
\AxiomC{$\Gamma_1 | \underline{\Delta} \vdash^c  M : A \sep \underline{B} $}
\AxiomC{$\Gamma_2 \vdash^v V : A$}
\RightLabel{$\sep Elim$}
\BinaryInfC{$\Gamma_1 \fatsemi \Gamma_2 | \underline{\Delta} \vdash^c M  \textit{@} V : \underline{B}$}
\end{prooftree}

\begin{prooftree}
\AxiomC{$\Gamma_1 \fatsemi (x : A) | \underline{\Delta} \vdash^c  M : \underline{B}$}
\AxiomC{$\Gamma_2  \vdash^v N : A$}
\RightLabel{$\sep \beta$}
\BinaryInfC{$\Gamma_1 \fatsemi \Gamma_2 | \underline{\Delta} \vdash^c (\alpha x : A.\; M) \textit{@} N = M[N/x] : \underline{B}$}
\end{prooftree}

\begin{prooftree}
\AxiomC{$\Gamma \fatsemi (x : A) | \underline{\Delta} \vdash^c  M : \underline{B}$}
\AxiomC{$x \notin FV(M)$}
\RightLabel{$\sep \eta$}
\BinaryInfC{$\Gamma | \underline{\Delta} \vdash^c (\alpha x : A.\; M \textit{@} \;x) = M : A \sep \underline{B}$}
\end{prooftree}


\subsection{F}\footnote{following \cite{Max-CT-W23}}
\begin{prooftree}
\AxiomC{$\Gamma \vdash^v M : A$}
\RightLabel{$\textbf{ret} Intro$}
\UnaryInfC{$\Gamma | \cdot \vdash^c : \textbf{ret} M : F \underline{A}$}
\end{prooftree}

\begin{prooftree}
\AxiomC{$\Gamma_1 | \underline{\Delta} \vdash^c M : F A$}
\AxiomC{$\Gamma_2 ; x : A | \cdot \vdash^c N : \underline{B}$}
\RightLabel{$\textbf{ret} Elim(;)$}
\BinaryInfC{$\blue{\Gamma_1 ; \Gamma_2} | \underline{\Delta} \vdash^c x \leftarrow M ; N : \underline{B}$}
\end{prooftree}

\begin{prooftree}
\AxiomC{$\Gamma_1 | \underline{\Delta} \vdash^c M : F A$}
\AxiomC{$\Gamma_2 \fatsemi x : A | \cdot \vdash^c N : \underline{B}$}
\RightLabel{$\textbf{ret} Elim(\fatsemi)$}
\BinaryInfC{$\blue{\Gamma_1 \fatsemi \Gamma_2} | \underline{\Delta} \vdash^c x \leftarrow M ; N : \underline{B}$}
\end{prooftree}

\begin{prooftree}
\AxiomC{$\Gamma_1 \vdash^v V : A$}
\AxiomC{$\Gamma_2 ; x : A | \cdot \vdash^c M : \underline{B}$}
\RightLabel{$\mathbf{ret} \; \beta(;)$}
\BinaryInfC{$\blue{\Gamma_1 ; \Gamma_2}  | \cdot \vdash^c (x \leftarrow \mathbf{ret} V ; M) = M[V/x] : \underline{B}$}
\end{prooftree}

\begin{prooftree}
\AxiomC{$\Gamma_1 \vdash^v V : A$}
\AxiomC{$\Gamma_2 \fatsemi x : A | \cdot \vdash^c M : \underline{B}$}
\RightLabel{$\mathbf{ret} \; \beta(\fatsemi)$}
\BinaryInfC{$\blue{\Gamma_1 \fatsemi \Gamma_2}  | \cdot \vdash^c (x \leftarrow \mathbf{ret} V ; M) = M[V/x] : \underline{B}$}
\end{prooftree}

\begin{prooftree}
\AxiomC{$\Gamma | \underline{\Delta} \vdash^c N : F A$}
\AxiomC{$\Gamma | \bullet : F A \vdash^c M : \underline{B}$}
\RightLabel{$\mathbf{ret} \; \eta$}
\BinaryInfC{$\Gamma | \underline{\Delta} \vdash^c M[N] = (x \leftarrow N ; M[\mathbf{ret}x]) : \underline{B}$}
\end{prooftree}

\section{Structural rules}
see page 20 of  https://link.springer.com/book/10.1007/978-94-017-0091-7

\begin{prooftree}
\AxiomC{$\Gamma \vdash^v M : A$}
\AxiomC{$\Upsilon(x : A) \vdash^v N : B$}
\RightLabel{$Cut$}
\BinaryInfC{$\Upsilon(\Gamma) \vdash^v N[M/x] : B$}
\end{prooftree}
is this $\string^$ needed?

\begin{prooftree}
\AxiomC{$\Gamma(\Upsilon) \vdash^v M : A$}
\RightLabel{$Weakening\;$(for $;$)}
\UnaryInfC{$\Gamma(\Upsilon \;; \Upsilon') \vdash^v M : A$}
\end{prooftree}



\begin{prooftree}
\AxiomC{$\Gamma(\Upsilon \; ; \Upsilon') \vdash^v M : A$}
\RightLabel{$(\Upsilon' \cong \Upsilon) Contraction\;$(for $;$)}
\UnaryInfC{$\Gamma(\Upsilon) \vdash^v M[i(\Upsilon)/i(\Upsilon')] : A$}
\end{prooftree}
\\
$\cong$ is isomorphism of bunches
\\
$i(\Upsilon)$ denote an in order traversal of the identifiers in $\Upsilon$
    
\begin{prooftree}
\AxiomC{$\Gamma \vdash^v M : A$}
\RightLabel{(where $\Gamma \equiv \Upsilon) Exchange$}
\UnaryInfC{$\Upsilon \vdash^v M : A$}
\end{prooftree}
\\
$\equiv$ is a coherence equivalece defined by 
\begin{itemize}
    \item Commutative monoid equations for $;$
    \item Commutative monoid equations for $\fatsemi$
    \item Congruence: $\Upsilon \equiv \Upsilon' \implies \Gamma(\Upsilon) \equiv \Gamma(\Upsilon')$
\end{itemize}

\section{Questions}
\begin{itemize}
    \item I've only "bunched" the value context.. the computation context already has a linear restraint on usage.. but do we need to say something about enforcing separation..?
\end{itemize}
\bibliographystyle{acm}
\bibliography{CBPV-OSum/osumref}
\end{document}
